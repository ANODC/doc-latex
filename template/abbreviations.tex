\Abbreviations

В настоящем документе применяют следующие термины с соответствующими определениями:
\begin{description}
\item [API] (программный интерфейс приложения, интерфейс прикладного программирования) (англ. application programming interface, API) — набор готовых классов, процедур, функций, структур и констант, предоставляемых приложением (библиотекой, сервисом) или операционной системой для использования во внешних программных продуктах.
\item [АС] автоматизированные системы.
\item [АРМ] автоматизированное рабочее место.
\item [АС] автоматизированные системы.
\item [БД] база данных.
\item [ВМ] виртуальная машина.
\item [ИР] информационные ресурсы.
\item [ИТ] информационные технологии.
\item [НСД] несанкционированный доступ.
\item [ПО] программное обеспечение.
\item [СЗИ] средства защиты информации.
\item [СХД] система хранения данных.

\item [Частное облако] (англ. private cloud) — инфраструктура, предназначенная для использования одной организацией, включающей несколько потребителей (например, подразделений одной организации), возможно также клиентами и подрядчиками данной организации. Частное облако может находиться в собственности, управлении и эксплуатации как самой организации, так и третьей стороны (или какой-либо их комбинации), и оно может физически существовать как внутри, так и вне юрисдикции владельца.

\item [Публичное облако] (англ. public cloud) — инфраструктура, предназначенная для свободного использования широкой публикой. Публичное облако может находиться в собственности, управлении и эксплуатации коммерческих, научных и правительственных организаций (или какой-либо их комбинации). Публичное облако физически существует в юрисдикции владельца — поставщика услуг.

\item [Гибридное облако] (англ. hybrid cloud) — это комбинация из двух или более различных облачных инфраструктур (частных, публичных или общественных), остающихся уникальными объектами, но связанных между собой стандартизованными или частными технологиями передачи данных и приложений.

\item [Общественное облако] (англ. community cloud) — вид инфраструктуры, предназначенный для использования конкретным сообществом потребителей из организаций, имеющих общие задачи (например, миссии, требований безопасности, политики, и соответствия различным требованиям). Общественное облако может находиться в кооперативной (совместной) собственности, управлении и эксплуатации одной или более из организаций сообщества или третьей стороны (или какой-либо их комбинации), и оно может физически существовать как внутри, так и вне юрисдикции владельца.

\end{description}