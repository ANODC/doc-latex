% 1. Стиль и язык
\documentclass[utf8x, 14pt]{G7-32} % Стиль (по умолчанию будет 14pt) А4, шрифт 12пунктов
\usepackage{ifthen}
%\title{Руководство пользователя}
%\usepackage[utf8]{inputenc}%включаем свою кодировку: koi8-r или utf8 в UNIX, cp1251 в Windows
%\usepackage[english,russian]{babel}%используем русский и английский языки с переносами
\usepackage[hidelinks]{hyperref} %ссылки на рисунки в документе
\usepackage{graphicx} %хотим вставлять рисунки
%\usepackage{xcolor} %изменение цвета текста
\usepackage{listings} %вставка кода программ
\usepackage{ragged2e} % выравнивание по краям
\usepackage{titlesec}

\usepackage{url} % hyperref works too
\usepackage{csvsimple}
\begin{filecontents*}{users-list.csv}
\end{filecontents*}

\usepackage{array}
\usepackage{float}

% Путь к рисункам
\graphicspath{{img/}}

% отключение переносов слов
%\tolerance-1
%\pretolerance=10000

\newcommand{\MainDomain}{domain.name}
\newcommand{\MailDomain}{@mail.domain.name}
\newcommand{\CompanyName}{Comany Name, LLC}

\def\appendix{}

\begin{document}

\frontmatter % выключает нумерацию ВСЕГО; здесь начинаются ненумерованные главы: реферат, введение, глоссарий, сокращения и прочее.

% Стиль титульного листа и заголовки
\begin{titlepage}
\newpage
\clearpage\maketitle
\thispagestyle{empty}

\begin{center}
\CompanyName \\
\vspace{0.1cm}
\hrulefill
\end{center}
 

\vspace{8em}

\begin{center}
\Large Руководство по работе с системами *.\MainDomain
\end{center}

\vspace{2.5em}
 
\begin{center}
\textsc{\textbf{Руководство пользователя}}
\end{center}

\vspace{6em}
 

 
\vspace{\fill}

\begin{center}
Москва, \today
\end{center}

\end{titlepage}


%\maketitle %создает титульную страницу


%\begin{executors}
%\personalSignature{Исполнитель}{Луньков С.В.}
%\end{executors}


%\listoffigures                         % Список рисунков

%\listoftables                          % Список таблиц

%\NormRefs % Нормативные ссылки 
% Команды \breakingbeforechapters и \nonbreakingbeforechapters
% управляют разрывом страницы перед главами.
% По-умолчанию страница разрывается.

% \nobreakingbeforechapters
% \breakingbeforechapters

\tableofcontents

\printnomenclature % Автоматический список сокращений

\Introduction

В данном документе содержиться руководство по работе с ... системой \CompanyName.\\

\Abbreviations

В настоящем документе применяют следующие термины с соответствующими определениями:
\begin{description}
\item [API] (программный интерфейс приложения, интерфейс прикладного программирования) (англ. application programming interface, API) — набор готовых классов, процедур, функций, структур и констант, предоставляемых приложением (библиотекой, сервисом) или операционной системой для использования во внешних программных продуктах.
\item [АС] автоматизированные системы.
\item [АРМ] автоматизированное рабочее место.
\item [АС] автоматизированные системы.
\item [БД] база данных.
\item [ВМ] виртуальная машина.
\item [ИР] информационные ресурсы.
\item [ИТ] информационные технологии.
\item [НСД] несанкционированный доступ.
\item [ПО] программное обеспечение.
\item [СЗИ] средства защиты информации.
\item [СХД] система хранения данных.

\item [Частное облако] (англ. private cloud) — инфраструктура, предназначенная для использования одной организацией, включающей несколько потребителей (например, подразделений одной организации), возможно также клиентами и подрядчиками данной организации. Частное облако может находиться в собственности, управлении и эксплуатации как самой организации, так и третьей стороны (или какой-либо их комбинации), и оно может физически существовать как внутри, так и вне юрисдикции владельца.

\item [Публичное облако] (англ. public cloud) — инфраструктура, предназначенная для свободного использования широкой публикой. Публичное облако может находиться в собственности, управлении и эксплуатации коммерческих, научных и правительственных организаций (или какой-либо их комбинации). Публичное облако физически существует в юрисдикции владельца — поставщика услуг.

\item [Гибридное облако] (англ. hybrid cloud) — это комбинация из двух или более различных облачных инфраструктур (частных, публичных или общественных), остающихся уникальными объектами, но связанных между собой стандартизованными или частными технологиями передачи данных и приложений.

\item [Общественное облако] (англ. community cloud) — вид инфраструктуры, предназначенный для использования конкретным сообществом потребителей из организаций, имеющих общие задачи (например, миссии, требований безопасности, политики, и соответствия различным требованиям). Общественное облако может находиться в кооперативной (совместной) собственности, управлении и эксплуатации одной или более из организаций сообщества или третьей стороны (или какой-либо их комбинации), и оно может физически существовать как внутри, так и вне юрисдикции владельца.

\end{description} % Введение

\mainmatter % это включает нумерацию глав и секций в документе ниже

\chapter{Основная часть}
\label{sec:chapter_main}


\backmatter %% Здесь заканчивается нумерованная часть документа и начинаются ссылки и


\ifdefined\appendix
\chapter{Приложение}
\label{appendix:appendix}
 % Тут идут приложения
\fi


\end{document}

%%% Local Variables:
%%% mode: latex
%%% TeX-master: t
%%% End:
